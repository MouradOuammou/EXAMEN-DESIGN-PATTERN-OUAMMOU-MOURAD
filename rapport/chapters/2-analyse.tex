\chapter{Analyse des besoins}

\section{Besoins fonctionnels}

\begin{table}[H]
    \centering
    \caption{Besoins fonctionnels du projet}
    \label{tab:besoins}
    \begin{tabular}{|l|p{8cm}|}
    \hline
    \textbf{ID} & \textbf{Description} \\
    \hline
    BF1 & Implémenter le pattern Builder \\
    \hline
    BF2 & Implémenter le pattern Observer \\
    \hline
    BF3 & Implémenter le pattern Strategy \\
    \hline
    BF4 & Implémenter le pattern Adapter \\
    \hline
    BF5 & Intégrer AspectJ pour le logging \\
    \hline
    BF6 & Créer une interface JavaFX \\
    \hline
    \end{tabular}
\end{table}

\section{Cas d'utilisation}

% Figure 2.1 : Diagramme de cas d'utilisation
\begin{figure}[H]
    \centering
    % \includegraphics[width=10cm]{figures/use-cases}
    \caption{Diagramme de cas d'utilisation}
    \label{fig:usecases}
\end{figure}

\section{Contraintes}
\begin{itemize}
    \item Utiliser Java 17+
    \item Maven comme outil de build
    \item Tests unitaires obligatoires
\end{itemize}
