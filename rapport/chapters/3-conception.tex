\chapter{Conception du système}

\section{Architecture générale}

Le système est organisé en plusieurs couches:

\begin{itemize}
    \item \textbf{Couche présentation} : Interface JavaFX avec contrôleurs
    \item \textbf{Couche métier} : Patterns et services implémentés
    \item \textbf{Couche aspects} : Logging transversal avec AspectJ
\end{itemize}

\section{Diagramme d'architecture}

\begin{figure}[H]
    \centering
    \begin{verbatim}
    ┌─────────────────────────────────────┐
    │      Couche Présentation (UI)       │
    │      (JavaFX MainWindow)             │
    └──────────────┬──────────────────────┘
                   │
    ┌──────────────▼──────────────────────┐
    │    Couche Métier (Controllers)      │
    │ (Builder, Observer, Strategy, etc) │
    └──────────────┬──────────────────────┘
                   │
    ┌──────────────▼──────────────────────┐
    │      Couche Services                │
    │ (Document, Event, Sorting Services) │
    └──────────────┬──────────────────────┘
                   │
    ┌──────────────▼──────────────────────┐
    │    Design Patterns & Aspects        │
    │ (Builder, Observer, Strategy,       │
    │  Adapter, AspectJ Logging)         │
    └─────────────────────────────────────┘
    \end{verbatim}
    \caption{Architecture en couches du système}
    \label{fig:architecture}
\end{figure}

\section{Design Patterns utilisés}

\subsection{Builder Pattern}

Le Builder Pattern est utilisé pour la construction de documents avec options flexibles.

\begin{table}[H]
    \centering
    \caption{Structure du Builder Pattern}
    \label{tab:builder-structure}
    \begin{tabular}{|l|l|}
    \hline
    \textbf{Classe} & \textbf{Responsabilité} \\
    \hline
    DocumentBuilder & Constructeur fluide \\
    \hline
    DocumentBuilder.Document & Objet construit \\
    \hline
    \end{tabular}
\end{table}

\subsection{Observer Pattern}

Le pattern Observer permet la notification d'événements à plusieurs listeners.

\begin{verbatim}
EventManager
    ├─ subscribe(listener)
    ├─ unsubscribe(listener)
    └─ notify(event)
       └─ EventListener.update()
\end{verbatim}

\subsection{Strategy Pattern}

Le Strategy Pattern permet de sélectionner l'algorithme de tri à l'exécution.

\begin{verbatim}
DataProcessor
    └─ setSortingStrategy(strategy)
       ├─ BubbleSort.sort()
       └─ QuickSort.sort()
\end{verbatim}

\subsection{Adapter Pattern}

L'Adapter Pattern adapte l'interface legacy à l'interface moderne.

\begin{verbatim}
DataSourceAdapter implements ModernDataInterface
    └─ getData()
       └─ LegacyDataSource.retrieveData()
\end{verbatim}

\section{Tableau récapitulatif des patterns}

\begin{table}[H]
    \centering
    \caption{Résumé complet des design patterns}
    \label{tab:patterns-summary}
    \begin{tabular}{|l|l|p{4cm}|l|}
    \hline
    \textbf{Pattern} & \textbf{Classe} & \textbf{Responsabilité} & \textbf{Type} \\
    \hline
    Builder & DocumentBuilder & Construction flexible & Créationnel \\
    \hline
    Observer & EventManager & Gestion d'événements & Comportement \\
    \hline
    Strategy & DataProcessor & Sélection d'algorithme & Comportement \\
    \hline
    Adapter & DataSourceAdapter & Adaptation d'interface & Structurel \\
    \hline
    AspectJ & LoggingAspect & Logging transversal & Structurel \\
    \hline
    \end{tabular}
\end{table}

\section{Services implémentés}

\subsection{Service Layer}

Les services fournissent une abstraction sur les patterns:

\begin{table}[H]
    \centering
    \caption{Services implémentés}
    \label{tab:services}
    \begin{tabular}{|l|p{6cm}|}
    \hline
    \textbf{Service} & \textbf{Fonctionnalité} \\
    \hline
    DocumentService & Création et gestion de documents \\
    \hline
    EventService & Gestion centralisée des événements \\
    \hline
    SortingService & Tri avec mesure de performance \\
    \hline
    DataAdapterService & Adaptation de sources de données \\
    \hline
    PatternDemoService & Façade regroupant tous les services \\
    \hline
    \end{tabular}
\end{table}

\section{Context Singleton}

\texttt{AppContext} est implémenté comme un singleton pour donner accès aux services depuis n'importe où dans l'application:

\begin{lstlisting}[language=Java]
AppContext context = AppContext.getInstance();
PatternDemoService service = context.getPatternDemoService();
\end{lstlisting}

\section{Aspects AspectJ}

L'aspect LoggingAspect capture les appels de méthodes et mesure les performances:

\begin{table}[H]
    \centering
    \caption{Advices AspectJ implémentés}
    \label{tab:aspects}
    \begin{tabular}{|l|l|l|}
    \hline
    \textbf{Advice} & \textbf{Pointcut} & \textbf{Action} \\
    \hline
    @Before & patternMethods() & Logging d'entrée \\
    \hline
    @AfterReturning & patternMethods() & Logging de sortie \\
    \hline
    @Around & serviceMethods() & Mesure de temps \\
    \hline
    \end{tabular}
\end{table}
