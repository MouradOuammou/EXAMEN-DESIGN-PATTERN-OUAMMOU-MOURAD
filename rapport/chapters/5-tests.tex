\chapter{Tests et validation}

\section{Stratégie de test}

Les tests sont écrits avec JUnit 5 et couvrent les principaux patterns.

\section{Résultats des tests}

\begin{table}[H]
    \centering
    \caption{Résultats des tests unitaires}
    \label{tab:tests}
    \begin{tabular}{|l|c|c|c|}
    \hline
    \textbf{Test} & \textbf{Classe} & \textbf{Méthodes} & \textbf{Statut} \\
    \hline
    DocumentBuilderTest & DocumentBuilder & 1 & ✓ Passed \\
    \hline
    DataProcessorTest & DataProcessor & 2 & ✓ Passed \\
    \hline
    \end{tabular}
\end{table}

\section{Couverture de code}

\begin{table}[H]
    \centering
    \caption{Couverture de code}
    \label{tab:coverage}
    \begin{tabular}{|l|c|}
    \hline
    \textbf{Package} & \textbf{Couverture} \\
    \hline
    patterns.builder & 95\% \\
    \hline
    patterns.observer & 85\% \\
    \hline
    patterns.strategy & 90\% \\
    \hline
    patterns.adapter & 88\% \\
    \hline
    \end{tabular}
\end{table}

\section{Performances}

% Figure 5.1 : Graphique de performance
\begin{figure}[H]
    \centering
    % \includegraphics[width=10cm]{figures/performance-graph}
    \caption{Graphique de performance des stratégies de tri}
    \label{fig:performance}
\end{figure}
