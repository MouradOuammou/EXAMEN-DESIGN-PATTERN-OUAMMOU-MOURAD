\chapter{Annexes}

\section{Structure des fichiers}

\begin{lstlisting}
design-patterns-project/
├── pom.xml
├── src/
│   ├── main/java/com/university/
│   │   ├── app/
│   │   ├── patterns/
│   │   └── aspects/
│   └── test/java/com/university/
└── rapport/
    ├── main.tex
    └── chapters/
        ├── 0-couverture.tex
        ├── 1-introduction.tex
        ├── 2-analyse.tex
        ├── 3-conception.tex
        ├── 4-implementation.tex
        ├── 5-tests.tex
        ├── 6-conclusion.tex
        └── a-annexes.tex
\end{lstlisting}

\section{Commandes Maven utiles}

\begin{lstlisting}[language=bash]
# Compiler le projet
mvn clean compile

# Exécuter les tests
mvn test

# Lancer l'application
mvn javafx:run

# Générer le JAR
mvn package
\end{lstlisting}

\section{Configuration AspectJ}

AspectJ est configuré dans le pom.xml avec le plugin Maven aspectj-maven-plugin. Les aspects sont automatiquement compilés lors de la compilation du projet.

\section{Dépendances}

\begin{table}[H]
    \centering
    \caption{Dépendances Maven principales}
    \label{tab:dependencies}
    \begin{tabular}{|l|l|}
    \hline
    \textbf{Dépendance} & \textbf{Version} \\
    \hline
    javafx-controls & 21 \\
    \hline
    javafx-fxml & 21 \\
    \hline
    aspectjrt & 1.9.21 \\
    \hline
    junit-jupiter-api & 5.9.2 \\
    \hline
    slf4j-api & 2.0.7 \\
    \hline
    \end{tabular}
\end{table}
